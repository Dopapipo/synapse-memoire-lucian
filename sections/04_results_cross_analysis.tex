
\section{Results and Cross-Analysis}

\subsection{RQ1 – Metrics and Detection}

Static metrics (e.g., LOC, CBO, LCOM) are heavily used in detection tools like SonarQube. 
Borg et al. (2024) confirmed empirically that low "Code Health" metrics correlate with high defect rates. 
However, architectural smells often escape metric-based tools. Visual approaches (Yamashita, 2015) and architectural anomaly clusters (Kazman et al., 2015) complement quantitative detection, suggesting the need for hybrid strategies.

\subsection{RQ2 – Categorization of Smells}

Bamizadeh et al. (2021) propose over 65 categorized smells grouped in 7 families, including Bloaters, Object-Orientation Abusers, Dispensables, and Change Preventers. The distinction between implementation-level and architectural-level smells proves fundamental for effective management. Typologies support targeted remediation strategies based on impact scope and criticality.

\subsection{RQ3 – Prioritization and Refactoring}

SonarQube assigns a "technical debt cost" per smell, enabling localized prioritization. 
Kazman et al. (2015) go further by estimating the ROI of architectural refactoring in real-world systems, linking structure to maintenance costs. 
Borg et al. (2024) confirm that proactive refactoring reduces defect propagation. 
However, no unified tool currently bridges the implementation–architecture divide. The vision of a global debt-aware platform is emerging.

\subsection{RQ4 – Practices and Long-Term Impact}

Clean Code, TDD, and continuous refactoring reduce smell accumulation and help maintain system health. Books like \textit{The Pragmatic Programmer} and \textit{Tidy First?} (Fowler, 2021) support a mindset of sustainable development. 
Beyond tools, organizational culture (e.g., DMS, onboarding practices) plays a crucial role in maintaining agility. 
For example, integrating SonarQube into company pipelines not only improves code quality but also simplifies onboarding for junior developers, yielding a higher human and productivity ROI.

\subsection{Cross-Analysis Table}

\begin{table}[H]
\centering
\begin{tabular}{|c|p{6cm}|p{6cm}|}
\hline
\textbf{RQ} & \textbf{Main Insight} & \textbf{Practical Implication} \\
\hline
RQ1 & Metrics detect but don't explain root causes & Combine metrics with structural and historical analysis \\
\hline
RQ2 & Categorization enables tailored refactoring & Design context-sensitive remediation strategies \\
\hline
RQ3 & Prioritization is stronger when linked to architectural ROI & Align technical decisions with business outcomes \\
\hline
RQ4 & Sustainable practices amplify code quality benefits & Culture matters more than tools alone \\
\hline
\end{tabular}
\caption{Synthesis of findings and implications across research questions}
\end{table}

\subsection{Summary}

A multi-level, multi-modal approach is needed to manage code smells and technical debt effectively. 
While tools like SonarQube offer valuable static insights, they must be embedded in organizational practices and combined with strategic thinking to be impactful. 
Agility, in this context, is not a methodology but a mindset — and lucidity, through introspection and iteration, becomes a driver for both individual and systemic improvement.
