
\section{Discussion and Recommendations}

\subsection{Bridging Theory and Practice}

This work confirms that managing code smells and technical debt is not just a technical issue, but an organizational and cultural challenge. Metrics help surface problems, but do not explain them. Tools can detect, but not prioritize with full context. Refactoring has measurable benefits — but only if the organization is ready to act.

The reflexive dimension of this review helped explore the tension between agile principles and actual practice. Lucidity becomes a core skill: the ability to pause, observe, interpret and iterate with intention.

\subsection{What Developers Can Do}

\begin{itemize}
    \item Regularly monitor SonarQube results, not as warnings but as a conversation with the code
    \item Learn to distinguish smells that are temporary from those that anchor deeper debt
    \item Practice ``Tidy First'' to clean up locally before starting new features
    \item Maintain a personal or team-based code journal to trace technical decisions
\end{itemize}

\subsection{What Architects and Tech Leads Can Do}

\begin{itemize}
    \item Schedule structural reviews (not only code reviews) monthly or per release
    \item Make architectural smells visible (via tools or visualizations)
    \item Track debt clusters and correlate them with team velocity or bugs
    \item Prioritize architectural refactorings using ROI arguments à la Kazman et al.
\end{itemize}

\subsection{What Teams Can Build}

\begin{itemize}
    \item A shared smell glossary tailored to the team's style and stack
    \item A light DMS-like ritual (weekly or monthly) to reflect on code quality + organizational health
    \item A habit of logging micro-decisions, like why code was duplicated “just this once”
    \item A system to surface low-code-health zones to new team members for better onboarding
\end{itemize}

\subsection{Long-Term Cultural Levers}

\begin{itemize}
    \item Normalize talking about debt — not as shame, but as evolution
    \item Combine agility with introspection: ritualize review moments, embrace feedback loops
    \item Integrate tooling (e.g., SonarQube) into onboarding flows to reduce mental load
    \item Encourage storytelling: refactoring isn't just cleaning, it's caretaking
\end{itemize}

\subsection{Limits and Future Work}

This study focused primarily on static smells and documented debt. Other dimensions (runtime smells, legacy rewrites, or non-code debt like outdated docs) were not deeply explored. Moreover, while the reflexive method enhanced the clarity of insights, it remains a subjective lens — further studies could enrich it with user experiments, team feedback, or codebase evolution timelines.

\subsection{A Developer’s Cultural Awakening}

This memory was not just an academic exercise — it became a personal act of refactoring.

Coming from limited but deeply intentional development experience, I’ve only worked in environments where care was culture: clean code, domain-driven design, automated pipelines, and collective rituals. My exposure has been brief — a 3-month internship and a 6-month apprenticeship — but both were deeply rooted in software craftsmanship.

In fact, I systematically refused or self-sabotaged every offer that didn't resonate with that mindset. I treated interviews as rehearsals, not opportunities, unless I sensed the presence of real vision and cultural depth.

Writing this memory helped me understand that refactoring is not just about code — it is about self. In rushing to write, then slowing down to rewrite, I began to see the act of shaping knowledge as a mirror to shaping systems. Each rewrite, each re-structure was a way to locate myself more clearly: not just as a developer, but as a future tech lead, a reflective builder, a prototyping thinker.

This work allowed me to consolidate the few, strong intuitions I had — while also challenging them. I didn’t just refactor this document. I refactored my cognitive architecture. And I now understand that onboarding is not just a technical step — it’s the first cultural rite of passage. What we choose to transmit, or hide, speaks volumes about what we value. And in that sense, refactoring becomes an act of care — not only for the system, but for those who will inherit it.
