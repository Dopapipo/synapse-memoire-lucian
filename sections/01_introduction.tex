
\section{Introduction}

\subsection{Motivation}

My interest in software development is deeply rooted in the idea of craftsmanship. I’ve always viewed code not just as instructions, but as a living structure — something to care for, to maintain, and to shape. During my professional experiences, I encountered moments where technical debt disrupted expected delivery, and I began to reflect on the true cost of messy code.

This memory explores how code smells relate to technical debt, and how lucidity — both technical and personal — can help us move forward.

\subsection{Main Research Problem}

How can we produce higher quality code by understanding, categorizing, and reducing code smells?

We explore this problem through four research questions:
\begin{itemize}
    \item \textbf{RQ1}: How do software metrics contribute to identifying code smells?
    \item \textbf{RQ2}: How can we categorize code smells effectively?
    \item \textbf{RQ3}: How can we use this categorization to prioritize and fix smells?
    \item \textbf{RQ4}: What are the long-term benefits (qualitative and economic) of managing code smells as technical debt?
\end{itemize}
